\bigskip
\setlist[itemize]{leftmargin=0pt}
\setlength{\parskip}{1pt}

\begin{itemize}

  \vbox{
  \item [\texttt{alteraciones}] 
    \begin{minipage}[t]{6cm}\flushleft
	    \textsf{(Número Entero)}
    \end{minipage}
    \hfill
    \begin{minipage}[t]{6cm}\flushright
      default: \texttt{0}
    \end{minipage}
    \vskip.2\baselineskip
    Cantidad de alteraciones en la armadura de clave.

Los números positivos representan sostenidos,
mientras que se refiere a bemoles con números negativos.
% (referir midi util doc, key signature)

}
\begin{minted}{yaml}
alteraciones: -2 # Bb

\end{minted}
  \vbox{
  \item [\texttt{canal}] 
    \begin{minipage}[t]{6cm}\flushleft
	    \textsf{(número entero)}
    \end{minipage}
    \hfill
    \begin{minipage}[t]{6cm}\flushright
      default: \texttt{1}
    \end{minipage}
    \vskip.2\baselineskip
    Número de Canal MIDI.

}
\begin{minted}{yaml}
canal: 3
    

\end{minted}
  \vbox{
  \item [\texttt{forma}] 
    \begin{minipage}[t]{6cm}\flushleft
	    \textsf{(lista de cadenas de caracteres)}
    \end{minipage}
    \hfill
    \begin{minipage}[t]{6cm}\flushright
      default: \texttt{[ None ]}
    \end{minipage}
    \vskip.2\baselineskip
    Estructura de la unidad.
Lista de unidades referidas a ser secuenciadas.

Cada elemento corresponde a un miembro de la paleta de unidades.

}
\begin{minted}{yaml}
forma: ['A', 'B']

\end{minted}
  \vbox{
  \item [\texttt{metro}] 
    \begin{minipage}[t]{6cm}\flushleft
	    \textsf{(cadena de caracteres)}
    \end{minipage}
    \hfill
    \begin{minipage}[t]{6cm}\flushright
      default: \texttt{4/4}
    \end{minipage}
    \vskip.2\baselineskip
    Clave de compás.

Cantidad de pulsos por compás sobre la relación de subdivisión.

}
\begin{minted}{yaml}
metro: 4/4

\end{minted}
  \vbox{
  \item [\texttt{modo}] 
    \begin{minipage}[t]{6cm}\flushleft
	    \textsf{(Número Entero)}
    \end{minipage}
    \hfill
    \begin{minipage}[t]{6cm}\flushright
      default: \texttt{0}
    \end{minipage}
    \vskip.2\baselineskip
    Modo de la escala.

El número 0 indica que se trata de una escala mayor, mientras que el 1 representa tonalidad menor.

}
\begin{minted}{yaml}
modo: 1 # menor

\end{minted}
  \vbox{
  \item [\texttt{registracion}] 
    \begin{minipage}[t]{6cm}\flushleft
	    \textsf{(lista de enteros)}
    \end{minipage}
    \hfill
    \begin{minipage}[t]{6cm}\flushright
      default: \texttt{[ 1 ]}
    \end{minipage}
    \vskip.2\baselineskip
    Conjunto de intervalos a ser indexados por el puntero de altura.

}
\begin{minted}{yaml}
registracion: [ 
  -12,-10, -9, -7, -5, -3, -2,
    0,  2,  3,  5,  7,  9, 10,
   12, 14, 15, 17, 19, 21, 22,
   24
]

\end{minted}
  \vbox{
  \item [\texttt{reiterar}] 
    \begin{minipage}[t]{6cm}\flushleft
	    \textsf{(número entero)}
    \end{minipage}
    \hfill
    \begin{minipage}[t]{6cm}\flushright
      default: \texttt{1}
    \end{minipage}
    \vskip.2\baselineskip
    Repeticiones, cantidad de veces q se toca esta unidad. 

Esta propiedad no es transferible, no se sucede
(de lo contrario se reiteraran los referidos).

}
\begin{minted}{yaml}
reiterar: 3

\end{minted}
  \vbox{
  \item [\texttt{transponer}] 
    \begin{minipage}[t]{6cm}\flushleft
	    \textsf{()}
    \end{minipage}
    \hfill
    \begin{minipage}[t]{6cm}\flushright
      default: \texttt{0}
    \end{minipage}
    \vskip.2\baselineskip
    Transponer puntero de intervalo.

Ajuste de alturas, pero dentro de la registración.

}
\begin{minted}{yaml}
transponer: 1

\end{minted}
  \vbox{
  \item [\texttt{transportar}] 
    \begin{minipage}[t]{6cm}\flushleft
	    \textsf{(número entero)}
    \end{minipage}
    \hfill
    \begin{minipage}[t]{6cm}\flushright
      default: \texttt{0}
    \end{minipage}
    \vskip.2\baselineskip
    Ajuste de alturas en semitonos.

}
\begin{minted}{yaml}
transportar: 60 # C

\end{minted}

\end{itemize}

\setlist[itemize]{leftmargin=11pt}
\setlength{\parskip}{6pt plus 2pt minus 1pt}