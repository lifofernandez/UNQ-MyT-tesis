\begin{tikzpicture}[node distance = 2cm, auto]

\tikzstyle{circulo} = [
	ellipse, 
	draw, 
    %fill=red!20, 
	minimum height=4em,
	text centered, 
	node distance=2cm,
   font=\bfseries
]

\tikzstyle{block} = [
	rectangle, 
	draw, 
	%fill=blue!20, 
	text width=7em, 
	text centered, 
	minimum height=4em,
	node distance=2cm,
]

\tikzstyle{line} = [
	draw,
	-latex',
]

\node [circulo]              (ana) {Análisis};
\node [circulo, text width=6em,below of=ana](dis) {Diseño de Gramática};
\node [circulo, text width=7em, below of=dis](dev) {Desarrollo del Contexto};
\node [circulo, below of=dev](doc) {Documentación};
\node [circulo, below of=doc](dep) {Publicación};

%\path [line] (ana) -- (dis) -- (dev) -- (doc) -- (dep);
\draw[densely dotted] (ana) -- (dis);
\draw[densely dotted] (dis) -- (dev);
\draw[densely dotted] (dev) -- (doc);
\draw[densely dotted] (doc) -- (dep);

\node [block, 
    right of=ana,
	node distance=4cm,
](boc) { 
      Boceto de Gramática
};

\node [block, 
    below of=boc,
](pro) { 
      Prototipo de Entorno
};

\path [line] (ana) -- (boc) -- (pro) -- (dis);

\node [block, 
    left of=dis,
	node distance=4cm,
](def) { 
   Definicion de Vocabulario y Jerarquías
};

\path [line] (dis) -- (def) |-  (dev) ;

\node [block, 
    right of=dev,
	node distance=4cm,
](per) { 
     Evaluación y desarrollo de herramientas
};

\node [block, 
    below of=per
](opt) { 
     Optimización
};

\path [line] (dev) -- (per) -- (opt) -- (doc);

\node [block, 
    left of=doc,
	node distance=4cm,
](fun) { 
   Indicaciones de Funcionamiento
};

\node [block, 
   below = 0.2cm of fun
](for) { 
   Revisión de Formateo 
};
\node [block, 
   below = 0.2cm of for
](not) { 
   Incorporación de Notas y Etiquetas 
};

\path [line] (doc) -- (fun) -- (for) -- (not) -| (dep) ;

\end{tikzpicture}

%%%%%%%%%%%%%%%%%%%%%%%%%%%%%%%%%%%%%%%%%%%%%%%%%%%%%%%%%%%%
% https://tex.stackexchange.com/questions/117063/tikz-center-node-below-2-other-nodes

%\begin{tikzpicture}
%  \node  (a) {a}  ;             
%  \node  (b) at (4,2) {b};
%  \path   (a) -- node {m} (b);
%  %  or \path (a) -- (b) node[midway]{m}; 
%  % or pos =.5  instead of midway
%\end{tikzpicture}
%
%\begin{tikzpicture} % with calc library
%  \node  (a) {a}  ;             
%  \node  (b) at (4,2) {b};
%  \node  at ($(a)!0.5!(b)$){MM};
%\end{tikzpicture}
%
%%%%%%%%%%%%%%%%%%%%%%%%%%%%%%%%%%%%%%%%%%%%%%%%%%%%%%%%%%%%%
%
%  - Boceto de Sintaxis | Prototipo en Perl
%
%  
%  - Gramática 
%  Vocabulario 
%
%  - Evaluación de Antecedentes 
%  Desarrollo del entorno      
%  Debugeo y Optimización      
%
%  - Formateo | Devel notes
%
%\begin{tikzpicture}[node distance=3cm,on grid]
%   \draw[help lines] (-6,-9) grid (6,1);
%  \node (top)          {TOP};
%
%  \node (node2)      [below = of top]   {B};
%
%  \node (pepe) [ above left = of node2]          { boceto };
%  \node (trueno) [ above right= of node2]          { prototipo };
%
%  \node (node1node2) [below left = of node2]          { AB };
%  \node (node2node3) [below right= of node2]          { BC };
%
%  \node (node1)      [above left  = of node1node2] {A};
%  \node (node3)      [above right = of node2node3] {C};
%
%  \node (node12node23) [below left= of node2node3]    { AB BC };
%
%%  \path (node1node2) -- node (bot) [text=red,below=3cm] {$\bot$} (node2node3);
%
%  %\draw (top)         edge (node1)
%  %                    edge (node2)
%  %                    edge (node3)
%  %\draw (node1node2)  edge (bot)
%  %                    edge (node1) 
%  %                    edge (node2);
%  %\draw (node2node3)  edge (bot)
%  %                    edge (node3) 
%
% \end{tikzpicture}
%%%%%%%%%%%%%%%%%%%%%%%%%%%%%%%%%%%%%%%%%%%%%%%%%%%%%%%%%%%%%


%\begin{tikzpicture}
%  [node distance=.8cm,
%  start chain=going below,]
%    % \node[punktchain, join] (intro) {Introduktion};
%    % \node[punktchain, join] (probf)      {Problemformulering};
%    % \node[punktchain, join] (investeringer)      {Investeringsteori};
%
%     \node[punktchain, join] (perfekt) {Boceto Gramatica};
%     \node[punktchain, join] (emperi) {Prótotipo Entorno};
%
%     \node (asym) [punktchain,
%    	node distance=4cm,
%]  {Sintaxis};
%
%     \begin{scope}[
%        start branch=venstre,
%        % We need to redefine the join-style to have the -> turn out right
%        every join/.style={->, thick, shorten <=1pt}
%      ]
%       \node[punktchain, on chain=going left, join=by {<-}] (risiko) {Léxico};
%     \end{scope}
%
%     \begin{scope}[
%        start branch=hoejre
%       ]
%       \node (finans) [punktchain, on chain=going right] {Contexto};
%     \end{scope}
%
%  \node[punktchain, join] (disk) { Optimizar };
%  \node[punktchain, join] (makro) { Documentación};
%  \node[punktchain, join] (konk) {Publicación};
%
%  %%  Connect (finans) with (disk). We want it to have square corners.
%  \draw[|-,-|,->, thick,] (finans.south) |-+(0,-1em)-| (risiko.south);
%
%   Llaves 
%  % No. 1
%  \draw[tuborg] let
%    \p1=(risiko.west), \p2=(finans.east) in
%    ($(\x1,\y1+2.5em)$) -- ($(\x2,\y2+2.5em)$) node[above, midway]  {Desarrollo};
%
%  %% No. 2
%  \draw[tuborg, decoration={brace}] let \p1=(disk.north), \p2=(makro.south) in
%    ($(2, \y1)$) -- ($(2, \y2)$) node[tubnode] {Postpro};
%
%  %% No. 3
%  \draw[tuborg, decoration={brace}] let \p1=(perfekt.north), \p2=(emperi.south) in
%    ($(2, \y1)$) -- ($(2, \y2)$) node[tubnode] {Ensayos};
%  \end{tikzpicture}
     
